\documentclass[12pt]{article}
\usepackage{graphicx}
\usepackage[margin=0.5in]{geometry}
\linespread{1.2}
\usepackage[utf8]{inputenc}
\usepackage[russian]{babel}
\usepackage{titlesec}
\usepackage{enumitem}
\usepackage{amsmath}
\titlelabel{\thetitle.\quad}
\date{}
\begin{document}

\title{\textbf{Критерии оценки номеров}}

\begin{figure}
\centering
\includegraphics[width=0.35\textwidth]{logo}
\end{figure}

\maketitle

\section{Группа критериев «Техника»}
Группа критериев оценивающая различные аспекты владения реквизитом.
\footnote{Если артист использует несколько видов реквизита, то критерии Качество исполнения, Сложность, Разнообразие и Музыкальность оцениваются по всем видам реквизита в совокупности, а не только по основному реквизиту. Таким образом низкая техника владения дополнительным реквизитом может уменьшать баллы по этим критериям, а высокая -- наоборот улучшить.}

\subsection{Качество исполнения} 
Оценивается чистота исполнения элементов и связок, независимо от уровня их сложности.
\begin{tabbing}
0\hspace{0.8em} \= -- низкое\\
0,5 \>-- среднее\\
1 \>-- выше среднего\\
1,5 \>-- близкое к идеальному\\
2 \>-- идеальное
\end{tabbing}

\subsection{Сложность} 
Оценивается уровень сложности элементов, переходов, и связок преобладающих в номере.

\begin{tabbing}
0\hspace{0.8em}  \= -- ниже базового уровня\\
0,5 \>-- базовый уровень\\
1 \>-- базовый уровень с элементами средней сложности\\
1,5 \> -- нестабильный средний уровень\\
2 \>-- стабильный средний уровень \\
2,5 \>-- чуть выше среднего уровня \\
3 \>-- средний с преобладанием продвинутого\\
3.5 \>-- продвинутый уровень\\ 
4 \>-- продвинутый уровень с исключительно сложными трюками
\end{tabbing}


\subsection{Разнообразие} 
Оценивается то насколько разнообразны элементы по принадлежности к классам, и есть ли повторяющиеся элементы и связки.

\begin{tabbing}
0\hspace{0.8em}  \= -- однообразные элементы или повторяющиеся связки\\
0,5 \>-- разнообразные элементы, но повторяющиеся связки\\
1 \>-- разнообразные элементы и не повторяющиеся связки
\end{tabbing}


\subsection{Музыкальность} 
Оценивается связь между движением реквизита и музыкой.

\begin{tabbing}
0\hspace{0.8em}  \= -- движения практически не связаны с музыкой\\
0,5 \>-- движения связаны с музыкой лишь отчасти\\
1 \>-- движения в целом связаны с музыкой большую часть номера\\
1,5 \>-- движения полностью связаны с музыкой или чётко попадают в ритм\\
2 \>-- движения полностью связаны с музыкой, и музыкальные акценты и паузы хорошо обыграны
\end{tabbing}


\subsection{Универсальность} 
Оценивается использование в номере нескольких видов реквизита.

\begin{tabbing}
0\hspace{0.8em}  \= -- только один вид реквизита\\
0,5 \>-- несколько видов реквизита\\
\end{tabbing}


\subsection{Синхронность}
\textbf{Критерий для коллективов и дуэтов}. Оценивается слаженность работы артистов с реквизитом и между собой, синхронность исполнения элементов, зеркальность исполнения, а также построение различных совместных элементов.

\begin{tabbing}
0\hspace{0.8em}  \= -- отсутствует\\
0,5 \>-- минимальная синхронность \\
1 \>-- синхронность меньшую часть номера\\
1,5 \>-- синхронность половину номера\\
2 \>-- синхронность большую часть номера, но с ошибками\\
2,5 -- синхронность большую часть номера\\
3 \>-- идеальная синхронность
\end{tabbing}


\section{Группа критериев «Артистизм»}

\subsection{Образ} 
Оценивается подробность образа, правдоподобность действий, и соответствие внешнего вида, поведения, и эмоций выбранному персонажу.
\footnote{Образ может быть любым: простым, сложным, узнаваемым, или необычным. Важно помнить, что сложный образ труднее раскрыть, простой раскрыть проще, но он рискует показаться неинтересным, а слишком простой может быть принят за его отсутствие.}

\begin{tabbing}
0\hspace{0.8em}  \= -- не раскрыт, не понятен или не интересен \\
0,5 \>-- угадывается, но практически не раскрыт\\
1 \>-- узнаваем, но плохо отыгран или реализован не до конца\\
1,5 \>-- раскрыт, понятен, и интересен
\end{tabbing}


\subsection{Удержание внимания}
Оценивается способность вызывать и удерживать интерес и внимание зрителя.
\footnote{Удерживать внимание можно любым способом: актёрской игрой, техникой, хореографией, или эффектами}

\begin{tabbing}
0\hspace{0.8em}  \= -- постоянная потеря внимания\\
0,5 \>-- частая потеря внимания\\
1 \>-- редкая потеря внимания, но в целом номер интересен\\
1,5 \>-- удержание внимания в течение всего номера
\end{tabbing}


\subsection{Эмоциональная работа} 
Оценивается сценическое обаяние, способность поддерживать контакт со зрителем, передавать текущее настроение персонажа, или выражать эмоциональное настроение номера в целом.

\begin{tabbing}
0\hspace{0.8em}  \= -- отсутствует\\
0,5 \>-- присутствует, но слабо\\
1 \>-- присутствует, артист поддерживает эмоциональный контакт со зрителем\\ 
1,5 \>-- присутствует, артист находится в контакте со зрителем, заражает эмоциями и настроением
\end{tabbing}


\section{Группа критериев «Внешний вид»}

\subsection{Наличие костюма} 

Оценивается наличие и степень проработки костюма.

\begin{tabbing}
0\hspace{0.8em}  \= -- костюм отсутствует\\
0,5 \>-- костюм простой или стандартный, либо сложный, но не соответствующий образу\\
1 \>-- интересный, красивый, и хорошо проработанный костюм
\end{tabbing}


\subsection{Эстетичность} 
Оценивается аккуратность, опрятность, и практичность (не мешает ли костюм движениям артиста), а также подчёркивает ли костюм достоинства фигуры или скрывает недостатки.

\begin{tabbing}
0\hspace{0.8em}  \= -- не эстетичен, или не опрятен, или не практичен.\\
0,5 \>-- эстетичен
\end{tabbing}


\section{Группа критериев «Сценическое движение»}

\subsection{Пластика}
Оценивается гармоничное и эстетичное движение тела.

\begin{tabbing}
0\hspace{0.8em}  \= -- отсутствует\\
0,5 \>-- присутствует, но лишь от части\\
1 \>-- присутствует
\end{tabbing}


\subsection{Хореография} 
Движение тела под музыку, выстраивающее определённую форму или композицию.
\footnote{Не имеет значения, придерживается ли артист одного из существующих стилей танца или демонстрирует собственный, оригинальный стиль}

\begin{tabbing}
0\hspace{0.8em}  \= -- отсутствует или плохо исполнена\\
0,5 \>-- присутствует\\
1 \>-- присутствует, а движения тела и реквизита гармонично дополняют друг друга
\end{tabbing}


\subsection{Акробатика} 
Оценивается наличие и исполнение прыжков, кувырков, переворотов, стоек, балансов и т.п.

\begin{tabbing}
0\hspace{0.8em}  \= -- отсутствует или плохо исполнена\\
0,5 \>-- присутствует, но без контакта с реквизитом\\
1 \>-- присутствует в контакте с реквизитом
\end{tabbing}


\subsection{Работа с пространством} 
Оценивается использование пространства сцены, свободное перемещение, работа на разных уровнях. Для коллективов оценивается коллективная работа с пространством, включая перестроения.

\begin{tabbing}
0\hspace{0.8em}  \= -- отсутствует\\
0,5 \>-- присутствует, но не оправдана образом, хореографией, или техническими элементами\\
1 \>-- присутствует и оправдана
\end{tabbing}


\section{Прочие критерии}


\subsection{Идея номера}
Оценивается основной замысел, определяющий содержание номера.

\begin{tabbing}
0\hspace{0.8em} \=-- не ясна или отсутствует\\
0,5 \>-- ясна, но не раскрыта полностью или должным образом\\
1 \>-- полностью ясна, раскрыта, и реализована
\end{tabbing}


\subsection{Дополнительные средства}
Оцениваются спецэффекты, декорации, сценический реквизит, а также любые материальные средства украшения номера.

\begin{tabbing}
0\hspace{0.8em} \=-- отсутствуют, или несут неясную функцию, или мешают артисту\\
0,5 \>-- присутствуют, уместны, и украшают номер
\end{tabbing}


\subsection{Оригинальность}
Оцениваются любые оригинальные, авторские, или новаторские решения в технике исполнения, реквизите, спецэффектах, декорациях и т.п.

\begin{tabbing}
0\hspace{0.8em} \=-- отсутствует, ничего нового\\
0,5 \>-- реализовано несколько оригинальных идей\\
1 \>-- присутствует в большом количестве
\end{tabbing}


\subsection{Штрафные баллы}
Если по вине артиста, появляются очевидно не запланированные обстоятельства, которые могут причинить или причинили вред здоровью артиста или зрителей, то судьи могу начислять штрафные баллы до --1 за потенциальную опасность, и до --2 за реальное происшествие. При оценке потенциальной угрозы, важно отличать просто опасный трюк, который прошёл по плану, и грубые нарушения техники безопасности. В частности, штрафные баллы начисляются за такие потенциальные угрозы, как:


-- Керосин на лице.

-- Бризинг с бутылкой топлива в руке.

-- Брызги или пламя в зрителей.

-- Не своевременное возгорание запасного реквизита или пиротехники.

-- Очевидно неконтролируемое горение реквизита или декораций.

\vspace{1em}
Напротив, штрафные баллы за потенциальную угрозу не начисляются в таких случаях как:
\vspace{1em}

+ Бёрнауты или иное контролируемое и безопасное стряхивание топлива с горящего реквизита.

+ Успешно исполненные, но опасные трюки, такие как глотание и выдувание огня, акробатика с огнём, использование огнемётов и т.п..

+ Рядовые падения реквизита.


\vspace{4em}
\hfill Версия от  \today
\end{document}
